% Chapter Template

\chapter{Literature Review} % Main chapter title

\label{Chapter2} % Change X to a consecutive number; for referencing this chapter elsewhere, use \ref{ChapterX}

%----------------------------------------------------------------------------------------
%	SECTION 1
%----------------------------------------------------------------------------------------

\section{Existing Literature }
Earlier studies on commodity market focus on price discovery. One dimensional model that assumes Brownian motion and constant convenience rate was first developed by Merton~\parencite{} in early 70's, followed by a series of mean reverting models which considered as better price forecaster. Popular models included Geometric Ornstein-Uhlenbeck model by Dixit and Pindyck~\parencite{dixit1994} which has an additional drift term to revert the price back to equilibrium level. Some models are further developed to include a convenience yield as a stochastic state such as in Gibson and Schwartz\parencite{Gibson1990}, where the parameters are estimated by weekly futures price. Later, Schwartz and Smith\parencite{smith2000} modified it to a two factor model that captures both long term mean and short term fluctuations in the prices. The long term mean is assumed to follow a geometric brownian motion with a drift term reflecting current market situation, and the short time derivation is modelled by an Ornstein-Uhlenbeck process. Due to the poor fitting of two factor models on commodity futures with longer maturity as well as the growing liquidity of longer maturity contracts, a three factor model is proposed again by Schwartz in 1997 under stochastic connivence and instantaneous interest rate following a mean reverting process as in Vasicek model~\parencite{vasicek1977}. Risk premia 
are assumed to be constant, parameters are extracted from historical data and through Kalman filter. More developments on the three factor model can be seen in paper by Hilliard and Reis~\parancite{hilliard_reis_1998} where the underlying spot prices follow a jump diffusion process~\parencite{bates1991} with non-zero mean jump to allow large discrete jumps. In addition, an equilibrium connivence yield diffusion and an arbitrage free interest rate diffusion are assumed. These models mentioned above seek to model the evolution of spot prices and then price the futures under the arbitrage free framework, finally the forward curve is derived from the future process. Because of the spot price are traditional observable and the most liquid, we adopt the same approach to model the forward curve and smile of volatility in commodity market and hence produce the feature prices and option prices quoted on futures. We follow the work done by Pallavicini~\parencite{Pallavicini2018} at el in which a local-volatility one-factor mean reverting process with affine drift is defined to model marginal density of the future prices and then a quick and robust calibration algorithm to all quoted options. Additionally, a stochastic local volatility dynamic is introduced to allow forward curve and smile structure for each future. The advantage of this model compared to traditional multi-factor stochastic volatility model is that the calibration is quick and robust such that it is applicable to partitioners on trading desks.
%-----------------------------------
%	SUBSECTION 1
%-----------------------------------
\subsection{Methodology}
The model of our interest is by Pallavicini at el.~\parencite{Pallavicini2018}. Related literature in the similar topic can be found in Pilz and Schl�gl~\parencite{}, Albani et al. \parencite{} and Chiminello~\parencite{}. However, none of them discussed 'Smile' effect related to strike prices is not discussedAmong these models, the first one considered a different stochastic process for each future price together with a designated  local volatility factor. Stochastic interest rates are incorporated for the forward structure in the futures prices. The second model introduces a distinct local volatility for each future price and a complete set of option quotes is simulated from existing ones to facilitate the calibration process. The last model considers a class of local volatility surfaces and assumes the future prices are diffusion process driven by such volatility surface without drift. In this report we adopt the the model described in \parencite{Pallavicini2018}. We define a two-step  calibration procedure. First, we introduce a linear local volatility model with an affine drift term~\parencite{} to model the spot prices. The future prices are then readily available in a closed form.  Option prices and futures prices are required to calibrate the stochastic parameters in the linear model. First, we start with the calibration of the futures prices by solving a first order ODE explicitly. We then proceed to calibrate the local volatility function in the dynamics of the futures prices. To achieve a quick and robust calibration, we employ 

an iterative calibration strategy by guessing the optimal parameters based on the mismatch from the target and model implied quantity as discussed in~ \parencite{Reghai}. The last step involves the calibration of mean reversion speed in the futures prices dynamics by looking at liquid mid-curve options (MCO) or calendar spread options(CSO), similarly in the work by~\parencite{anderson}. To model the curve and skew dynamics, we look at the joint probability densities between multiple futures prices and the transition densities in each futures dynamics. In Pallavicini at el's model, they promote each futures price to follow a different SLV model, a new calibration method is developed in which a two dimensional model including the local volatility and correlation parameters is considered. The calibration of mean reversion and correlation parameters to market quotes can be
done by solving the corresponding bi-dimensional pricing PDEs. 
Finally, if we wish to also model the smile dynamics we need to introduce a stochastic volatility process as defined in \parencite{ren} with a three factor process and the calibration procedures can be derived henceforward. 
%-----------------------------------
%	SUBSECTION 2
%-----------------------------------
