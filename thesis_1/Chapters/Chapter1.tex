% Chapter 1

\chapter{Introduction} % Main chapter title

\label{Chapter1} % For referencing the chapter elsewhere, use \ref{Chapter1} 

%----------------------------------------------------------------------------------------

% Define some commands to keep the formatting separated from the content 
\newcommand{\keyword}[1]{\textbf{#1}}
\newcommand{\tabhead}[1]{\textbf{#1}}
\newcommand{\code}[1]{\texttt{#1}}
\newcommand{\file}[1]{\texttt{\bfseries#1}}
\newcommand{\option}[1]{\texttt{\itshape#1}}

%----------------------------------------------------------------------------------------
\section{Commodity Market Overview}
Commodity trading is as old as civilisation. It is believed that the first commodity market can be long dated back to between 4500 and 4000 BC in Sumer. Sumerians used clay writing tablets to document the amount of goods and delivery date of the trade. This ancient trading system reassembles to today's futures contracts in commodity market. Medieval Europe saw a prosperous growth in Commodity market, with the first stock exchange established in Amsterdam in 1530 which also operated as a market for the exchange of commodities. Nowadays, the commodity trading is no longer restricted to the physical format. In fact, among over 50 major commodities exchanges worldwide, purely financial transactions has been increasingly outnumbering physical trades particularly in energy market. 

A commodity market is referred to a physical or virtual platform that trades in primary economic sector. Modern commodity trading includes the form of spot trading and derivative trading. In spot trading market, physical delivery of goods and payment are involved between the market participants. The prices in this market is a reflection of the current or very near term market equilibrium of demand and supply. 

The derivatives market, on the other hand, provides trading platform of derivative contracts written on the spot price. These derivatives take form of forwards, futures and options on futures with futures being the most traded one. Commodity features prices reflect a collective review from both buy side and sell side of the future market supply and demand. Commodity derivatives are secured by physical assets or commodities. Hence unlike bonds and stocks, commodities are valued not based on the profitability or cash flows but rather on the estimated future prices implied by the market demand and supply of the physical item. Due to this nature, commodity market remains the most volatile asset class. Crude oil has seen quarterly volatility surge from 12.63\% to over 90\% since 1983. During financial crisis, in particular, the price crashed almost 500\% in less than half a year--the WTI crude oil prices reached its peak of US\$147.27 on 11 July 2008 and fell to US\$30.28 a barrel on 23 December the same year. Even more evidently, the price of WTI crude oil features contract plunged into negative in May 2020 settled at US-\$37.63 a barrel for the first time in history due to the Covid-19. Similar examples include: 1970's oil crises\parencite{olicrises}and 1980's oil glut\parencite{oliglut}. This reflects the fact that alongside with mother nature, geopolitics and speculations also influence the global demand and supply of commodities greatly. Because of the volatile nature of the commodity, many institutional users of commodities confronted with rising raw material prices may wish to lock in low prices to avoid risks in future prices. Airline companies, for example, secure a massive amount of fuels at a reasonable price by entering a future contract.  A future contract is a standardised legal agreement to buy or sell a commodity at a predetermined price often with agreed quantity and quality and a specific time in the futures. At this future date, either physical delivery of the commodity or cash settlement will take place. In practice, physical delivery only occurs in a minority of contracts since it can be very costly taking into account the delivery and storage expense. Most commodity futures traders often offset their contracts before expiry date. This is normally done by purchasing a covering position, Nymex crude futures contract uses this method of settlement upon expiration. To maintain the same risk position beyond the initial expiration, some traders also choose to roll over the contract--switch from the near expiration contract to one further-out month. The original contract has to be closed meaning that the loss or gain is required to be settled before entering into a new one. In case of physical delivery, the position has to be closed before first notification date and last notification date for a cash settled contract. 

Many financial futures based on market indexes are cash settled, for example, the popular E-minis which track the S\&P 500 market index was launched in 1997 and traded on the Chicago Mercantile Exchange (CME) via their Globex electronic trading platform. Cash settled futures contracts are preferred by a lot of the investors with its two major advantages\parencite{lien_tse_2002}. First, a cash settled contract ensures the convergence between spot price and futures price at expiration which enhances its power as a hedging instrument. Second, cash settlement reduces the market manipulation such as cornering and squeezing.  

Commodity market has continued gaining traction in the recent decades, many portfolio managers add commodity derivatives as an asset class. Basu and Gavin\parencite{basu_gavin_2011} proposed two possible explanations for the surge in trading commodity derivatives. With first one being greater appetite for risky asset in general and the second explanation is a mistaken notion that an investment in commodity futures can be used to hedge equity risk.



%----------------------------------------------------------------------------------------








\subsection{Market Convention}
Commonly traded commodities are split into four categories: energy, metal, livestock and meat, Agriculture. Energy commodities include crude oil, heating oil, natural gas, and gasoline. Prices of crude oil is extremely sensitive to consumer and investor sentiment. As such the crude oil market is very volatile to geopolitics and extreme climate. With West Texas Intermediate (WTI) and Brent North Sea Crude(Brent Crude) being the most popular traded ones on the market, Brent crude's price is the benchmark for African, European, and Middle Eastern crude oil, while WTI is he benchmark for North America. The futures contracts of WTI are listed on New York Mercantile Exchange (NYMEX) and the physical delivery happens in Cushing, Oklahoma\parencite{cme_future}.  Brent crude oil futures trade on the Intercontinental Exchange (ICE) and are traded globally with various delivery locations around the world. Both Brent Crude and WTI has a contract size with 1,000 barrels, with 134 maximum contracts for WTI and 96 for Brent Crude in the futures contract series are available for trading. 
Physical delivery often takes forms of inter-facility transfer through pipeline, in-line transfer or in-tank transfer where the crude oil remains physically stationary but the title is transferred. 


Metals are another important resources for production and investment purpose next to crude oil. The precious metals with active markets include: gold, silver, platinum.  Investors tend to invest in precious metals during time of market market volatility and uncertainty or period of high inflation to mitigate the risk. The three main factors in the metal market are: demand from China, technological  innovations and  institutional regulations. S\&P GSCI Precious Metals Index( \string^ SPGSI), S\&P GSCI Industrial Metals Select( \string^ SPGSINP) and UBS Bloomberg CMCI Industrial Metals Index Total Return ( \string^ UBM-SO).

The difference between spot and futures prices is generally called the basis. When the spot price is higher than the futures price, it is called backwardation, and when it is lower it is called contango. Backwardation and contango are also used to describe the relationship between two futures contracts of the same commodity.







\subsection{Motivation}
Increased volatility of commodity prices add great risk to economy and investment. For commodity investors, changing in price directly affects the \textit{marginal convenience yield}, an adjustment to the cost of carry in forward pricing formula under non-arbitrage argument which make backwardation pronounce. Secondly, price volatilities also increase the \textit{option premium}, i.e the opportunity cost of exercising the option instead of holding up for more information\parencite{pindyck}. This opportunity cost reinforces the backwardation in the first effect we listed\parencite{litzenberger}. 

Volatility dynamics are a key consideration in strategy formation for hedging, derivatives
trading, and portfolio optimisation. Commodity futures, in particular, add a valuable source of diversification benefits compared to traditional assets class for investors and portfolio managers. Because the futures market is highly liquid and volatile, derivatives on futures price have also gained more tractions. However, many trades and investments decisions in commodity derivatives market are not made under a correct pricing scheme. Especially for customised derivative contracts that are illiquid, opaque, or have little market depth or limited expirations, a pricing model for future prices that incorporates both the forward curve and smile dynamic is required. Such a model provide an accurate price discovery, option modelling and risk management for investors. In addition, a comprehensive understanding and modelling of futures dynamic also enables policy makers to re-evaluate the adequacy of current regulations and interventions to better improve the social welfare. 


\subsection{Objective}
This thesis concerns two closely related objectives. The first objective is to propose a method to simulate realistic futures price and quoted option prices on commodity market. Secondly, to present stochastic-local volatility model which incorporates both forward-curve and smile dynamics of the volatility. 







%-----------------------------------------------------------
\subsection{Methodology and Outline}


 
%----------------------------------------------------------------------------------------





%----------------------------------------------------------------------------------------



%----------------------------------------------------------------------------------------


